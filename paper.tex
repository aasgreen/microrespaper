\documentclass[reprint]{revtex4-1}
\usepackage{hyperref,graphics,float}

\begin{document}
\title{Using Micoresonators as frequency stabilizers for lasers}
\date{\today}
\author{Adam Green}
\affiliation{NIST, CU Boulder Physics}

\begin{abstract}
\begin{description}
\item[Background]
\item[Purpose] As the title said
\item[Method]
\item[Results] Currently unknown
\end{description}
\end{abstract}
\keywords{microresonators}

\maketitle

\section*{\label{sec:intro}Introduction}
This will contain the introduction. See scott's already very comprehensive notes

\section*{Characterizing The Resonators}
\subsection*{Construction}
Put in excel file with the resonator settings.
\subsection*{Q}
\subsubsection*{Unloaded vs Loaded Q}
Ujsing the input-output formalism~\cite{thesis:schliesser}, the transmission
of the cavity-taper coupling, on resonance, can be shown to be:
\[
T = \left( \frac{1-K}{1+K}\right)^2
,
\]
where $K=\frac{\tau_0}{\tau_{\mathrm{external}}}$ is the ratio of the 
intrinsic photon-lifetime, and the photon-lifetime due to the coupling
of the fiber-taper.

We also know that,
\[
\frac{1}{\tau_\mathrm{total}} = \frac{1}{\tau_\mathrm{0}}+\frac{1}{\tau_\mathrm{external}}
\]
and since we can directly measure the linewidth from our scope trace, we can calculate the unloaded $Q_0 = \omega \tau_0$.

Solving for $K$ gives:
\[
K = \left( \frac{1-T^{1/2}}{1+T^{1/2}}\right),
\],
and using this in our equation for $Q_0$ gives:
\begin{equation}
\label{eq:qnought}
Q_0 = \frac{2}{1+T^{1/2}}
\end{equation}
\subsection*{Frequency Drift}

\section*{Setup}
This will contain pictures and descriptions of our current 
setups.
\bibliographystyle{plain}
\bibliography{paper}
\end{document}
