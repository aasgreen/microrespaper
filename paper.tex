\documentclass[reprint]{revtex4-1}
\usepackage{hyperref,graphics,float}

\begin{document}
\title{Using Micoresonators as frequency stabilizers for lasers}
\date{\today}
\author{Adam Green}
\affiliation{NIST, CU Boulder Physics}

\begin{abstract}
\begin{description}
\item[Background]
\item[Purpose] As the title said
\item[Method]
\item[Results] Currently unknown
\end{description}
\end{abstract}
\keywords{microresonators}

\maketitle

\section*{\label{sec:intro}Introduction}
This will contain the introduction. See scott's already very comprehensive notes

\section*{Characterizing The Resonators}
\subsection*{Construction}
Put in excel file with the resonator settings.
\subsection*{Q}
\subsubsection*{Unloaded vs Loaded Q}
I am still having some trouble understanding how to extract the intrinsic Q from the loaded Q. Kippenburg's thesis and scott's notes talk about it a little, but I will need a little more theory background.
\subsection*{Frequency Drift}

\section*{Setup}
This will contain pictures and descriptions of our current 
setups.

\end{document}
